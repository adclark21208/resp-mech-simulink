\section{Introduzione}

% \noindent
\textcolor{blue}{
	\lipsum[1-2]
}

\section{Background}


SCRIVI QUALCOSA

\subsection{Analogia circuitale}

\begin{figure*}[t!]
	\begin{subfigure}{0.5\linewidth}
		\centering
		\small{
			\def\svgwidth{\linewidth}
			\input{circuit.pdf_tex}}
		\caption{}
		\label{fig:modello}
	\end{subfigure}\hfill
	\begin{subfigure}{0.5\linewidth}
		\centering
		\small{
			\def\svgwidth{0.7\linewidth}
			\input{lung.pdf_tex}}
		\caption{}
	\end{subfigure}\hfill
	\caption{Analogia circuitale della meccanica respiratoria \cite{khoo_physiological_2018} (a); Rappresentazione schematica della divisione del circuito polmonare in due contributi resistivi (vie aeree superiori e inferiori) e in due contributi capacitivi (compliance del polmone e della parete toracica), raffigurate anche la pressione alveolare e pleurica.}
\end{figure*}

Il circuito polmonare può essere analizzando facendo un'analogia con i circuiti elettrici.

In particolare è possibile fare un parallelismo tra il flusso d'aria e la corrente elettrica (flusso di cariche) vedendo e la pressione come la presenza di un potenziale elettrico.
Si rivede allora la resistenza meccanica come il rapporto tra l'incremento di pressione rispetto il flusso, analoga alla resisitenza elettrica. Similmente la compliance non è altro che il rapporto tra l'aumento di volume e l'aumento di pressione, in analogia elettrica è un condensatore.

Il sistema in \cref{fig:modello} è un modello di meccanica respiratoria che trascura la presenza di contributi inerziali (non ci induttanze) e considera la presenza di due compartimenti. Sono separate le vie aeree superiori, con il loro contributo resistivo $R_C$ dalle vie aeree inferiori $R_P$. I due compartimenti sono in serie tra loro ed in serie ai serbatoi d'aria, ovvero le capacità rappresentanti il contributo di compliance della parete $C_W$ e del polmone $C_L$. 
Tali contributi sono in serie proprio perchè il volume d'aria passante è lo stesso. 

A questo si aggiunge anche la capacità di shunt $C_S$ che tiene conto di diversi contributi quali lo spazio morto anatomico, la deformabilità delle vie aeree e la comprimibilità dell'aria. 

Si identificano allora anche le pressioni nei nodi. La pressione alle vie aeree $P_{aw}$, la pressione pleurica $P_{pl}$ e la pressione alveolare $P_A$. Chiaramente l'ingresso del sistema, dato dalla bocca e dalle cavità nasali, è rappresentato dalla pressione all'apertura delle vie aeree $P_{aO}$. 



\subsection{Risposta del sistema}

Il circuito in \cref{fig:modello} può essere descritto dalle seguenti equazioni:

\begin{equation}
	\footnotesize{
	\left\{\begin{array}{l}
		P_{a O}=Q R_{C}+\frac{1}{C_{S}} \int\left(Q-Q_{A}\right) \\
		\frac{1}{C_{s}} \int\left(Q-Q_{A}\right)=Q_{A} R_{P}+\left(\frac{1}{C_{L}}+\frac{1}{C_{W}}\right) \int Q_{A}
	\end{array}\right.}
\end{equation}

Si ottiene allora la funzione di trasferimento del sistema:

\begin{equation}
		\footnotesize{
\begin{aligned}
	H(s)&=\frac{Q(s)}{P_{a O}(s)}\\
	&=\frac{s^{2}+s \frac{1}{R_{P}}\left(\frac{1}{C_{S}}+\frac{1}{C_{e q}}\right)}{s^{2}\left(R_{C}\right)+s\left(\frac{R_{C}+R_{P}+\frac{R_{C} C_{S}}{C_{e q}}}{C_{S} R_{P}}\right)+\frac{1}{C_{e q} C_{S} R_{P}}}
\end{aligned}}
\end{equation}

Dove si esprime la serie delle capacità come ${1\over C_{eq}}={1\over C_{L}}+{1\over C_{W}}$. 

\subsection{Proprietà del sistema}

I coefficienti numerici vengono selezionati da \citeauthor{khoo_physiological_2018} \cite{khoo_physiological_2018}, sono riportati in \cref{tab:coefficienti}.


\begin{table}[!ht]
	\centering
	\begin{tabular}{|c|c|c|}
		\hline
		Parametro & Valore & Unità \\ \hline
		$R_C$ & 1 & H\textsubscript{2}O s / L \\ \hline
		$R_P$ & 0.5 & H\textsubscript{2}O s / L \\ \hline
		$C_L$ & 0.2 & L cm / H\textsubscript{2}O \\ \hline
		$C_W$ & 0.2 & L cm / H\textsubscript{2}O \\ \hline
		$C_S$ & 0.005 & L cm / H\textsubscript{2}O \\ \hline
	\end{tabular}
\caption{Coefficienti numerici per il sistema \cite{khoo_physiological_2018}}
\label{tab:coefficienti}
\end{table}

\section{Modellazione del sistema}

\subsection{Simulink}



\section{Conclusioni}


%\pagebreak
\section*{Disponibilità dei dati}

Il materiale è disponibile alla repository online del progetto: \url{https://github.com/mastroalex/resp-mech-simulink}

\subsection*{Codice}

\raggedbottom

\pagebreak
\printbibliography[title=Riferimenti]
%\section*{References}



